\documentclass[sigconf]{acmart}
\usepackage{polyglossia}
\usepackage{booktabs} % For formal tables

\begin{document}
\title{TITLE OF PAPER HERE}

\author{Benjamin Bush}
\affiliation{%
  \institution{Washington University in St. Louis}
  \streetaddress{1 Brookings Drive}
  \city{St. Louis}
  \state{Missouri}
  \postcode{63105}
}
\email{ben.bush@wustl.edu}



\begin{abstract}
Intelligent transportation relies on the Internet of Things (IoT), real-time cloud technologies, and
machine learning algorithms to manage traffic flow. Intelligent transportation systems (ITS)
require latency guarantees and data analytics solutions capable of ingesting and analyzing data in
real-time to produce timely signals that are suitable for the prevailing traffic conditions. As data
analytics engines migrate to virtualized environments, there is a need to preserve real-time
performance on these virtualized platforms. In this research, we introduce a model of an ITS that
predicts future traffic conditions from streaming data. This ITS is used as real-time workload in a
virtualized environment to empirically compare latency guarantees using the Xen hypervisor.
Experimental results show that the RTDS scheduler minimizes the latency of the streaming
application while maintaining real-time performance.
\end{abstract}



\keywords{RT-Xen, realtime systems, machine learning, LSTM}


\maketitle

\section{Introduction}

ITSs are perhaps the most anticipated smarty city services and have already seen widespread
adoption. The Sydney Coordinated Adaptive Traffic System (SCATS) is a fully adaptive urban
traffic control system that optimizes traffic flow and currently operates in more than 37,000
intersections worldwide (Citation). Optimizing traffic flow conditions not only shortens travel
times, but also can also reduce the carbon emissions generated from road vehicle activity
(Chong-White C., Millar M., Johnson F. and Shaw S. (2011). The evolution of the SCATS and
the environment study, Papers of the 34th Australasian transport research forum, Adelaide,
Australia.). Although there has been much research into the development of ITSs, most
approaches relied on mathematical equations and simulations. However, such approaches do not
encapsulate real world conditions such as weather, traffic accidents, and external events like
sports games (2-5 from here https://arxiv.org/ftp/arxiv/papers/1803/1803.02099.pdf). With the
advent of IoT, wireless sensor networks (WANs) and wireless sensor actuator networks
(WSANs) are being deployed to collect large-scale data of the physical world (Citation). Sensors
can collect data from cameras, inductor loops, GPS coordinates, etc. all in real-time. These
sensor networks record huge amounts data and then transmit this data to the cloud. Large-scale
analytics tools can then turn sensor data into decisions to optimize the operations of a city.

\section{Background}
\subsection{Traffic Flow Prediction}
\subsection{Xen}
\subsection{RT-Xen}
\section{Deep Learning for Traffic Flow Prediction}
\subsection{Dataset}
\subsection{LSTM}
\subsubsection{Training}
\section{Experimental Setup}
\section{Evaluation}
\section{Future Work}
\section{Conclusion}




\end{document}
